% Chapter Template

\chapter{Computer Science Adequacy} % Main chapter title

\label{Chapter4} % Change X to a consecutive number; for referencing this chapter elsewhere, use \ref{ChapterX}

https://www.fib.upc.edu/ca/estudis/graus/grau-en-enginyeria-informatica/pla-destudis/especialitats/computacio\\




Un graduat especialitzat en Computació haurà adquirit els coneixements necessàries per dissenyar sistemes informàtics complexos i crítics en termes d'eficiència, fiabilitat i seguretat. Des de la planificació dels vols d'un aeroport, o la verificació del funcionament d'un sistema de frenada ABS, fins al disseny de la interfície persona-màquina dels mòbils del futur. La coresponsabilitat social que obliga a exigir solucions cada cop més eficients, energèticament o econòmica per exemple, fa de l'informàtic amb aquestes habilitats un professional altament valorat en àmbits molt diversos. Per exemple, en àrees com la robòtica i l'optimització de processos a la indústria, els productes financers i la predicció a la banca, la planificació d'infraestructures a l'administració pública, l'experimentació científica i el tractament d'imatges en centres de recerca biomèdica, o la programació de jocs i aplicacions del web a la industria pròpiament informàtica.

La creixent exigència d'innovació front als nous reptes requereix de professionals entrenats per treballar amb rigor científic i que puguin integrar-se en equips multidisciplinars d'enginyers, científics o economistes. La vàlua de l'especialista en computació radica en la seva habilitat per innovar, i per detectar i garantir els requeriments crítics d'un sistema informàtic complex. Aquesta tendència en la nova indústria informàtica ve liderada per les firmes de més prestigi d'àmbit global.
