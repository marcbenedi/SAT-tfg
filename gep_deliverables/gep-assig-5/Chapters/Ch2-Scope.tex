% Chapter Template

\chapter{Scope} % Main chapter title

\label{Chapter2} % Change X to a consecutive number; for referencing this chapter elsewhere, use \ref{ChapterX}

\section{What and how?}

To achieve all the general objectives\ref{Chapter1} of the project, the following stages have been established:
\begin{itemize}
	\item Analyze, refactor\footnote{Code refactoring is the process of restructuring existing computer code—changing the factoring—without changing its external behaviour. \href{https://en.wikipedia.org/wiki/Code_refactoring}{(more)}} and test the existing code to have a solid base. 
	\item Add the functionality of representing \emph{PBF}.
	\item Study \href{http://tools.computational-logic.org/content/pblib.php}{PBLib} library to see which functionalities it has available to work with \emph{minimization}.
	\item Implement \emph{minimization} strategies.
	\item Study timeout strategies and implement them.
	\item Study and implement multithreading.
\end{itemize}

\section{Possible obstacles}

TODO: Possible obstacles that
may hinder the execution of the
project are briefly stated.

In this section, the possible obstacles and its solutions are exposed.

\subsection{Base project}
This project will be built on top of an existing one, as explained in \emph{Background section}\ref{Chapter1}. The existing project could be a source of bugs and other problems caused by not following an adequate methodology. For this reason and to solve possible issues, the first stage of the project will be focused on solving them.
\subsection{Schedule}
Due to the circumstances in which this project will be developed (Erasmus) possible delays could appear. To fix these circumstances, a realistic schedule with weekly communication will be planned. This will support a continuous development and detect as soon as possible delays. 

\subsection{PBLib}
One of the main requirements of this project, \emph{Pseudo-Boolean minimization}, is planned to be done with \emph{PBLib} library. It may be this library does not fit as expected with the project forcing to find a substitute. 

\subsection{Correctness}
As explained in \emph{Rigor and Validation}\ref{Chapter4}, correctness in this project is very important because of the context it is in. \\
Guarantee correctness could be hard and take more time than expected. If this happens, formal correctness could be delayed or reduced. 