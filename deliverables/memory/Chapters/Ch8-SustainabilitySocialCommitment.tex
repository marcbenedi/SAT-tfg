\chapter{Sustainability and Social Commitment}
\label{Chapter8}

\section{Sustainability Matrix}
In this section, the sustainability matrix is resumed according to the numbers described here\cite{MatrixPoints}.

\begin{table}[h]
	\centering
	\begin{tabular}{l|c|c|c|}
		\cline{2-4}
		& \multicolumn{1}{l|}{PPP} & Useful life & Risks \\ \hline
		\multicolumn{1}{|l|}{Environmental} & 7 & 20& -4\\ \hline
		\multicolumn{1}{|l|}{Economical} & 7 & 15 & -10\\ \hline
		\multicolumn{1}{|l|}{Social} & 8 & 15 & 0\\ \hline
		\multicolumn{1}{|l|}{Sustainability range} & \multicolumn{3}{c|}{58} \\ \hline
	\end{tabular}
	\caption{Sustainability matrix}
	\label{SustainabilityMatrix}
\end{table}


\section{Economic dimension}

\subsection{PPP}
The estimated budget of the project can be found in table \ref{TotalBudget}. The estimated budget is 13.997,48€. This number has been estimated taking into account the working hours of each role, the hardware and software used, indirect costs, contingency, and unforeseen events.
 
\subsection{Shelf life}
Nowadays, the no optimization of Pseudo-Boolean encodings implies that the problems are bigger and harder which causes a long execution and more consumption of resources. With the optimizations that this project will study, the final execution time could be reduced therefore the power needed to solve the problem which translates into a more reduced cost.

\subsection{Risks}
As exposed previously, some risks are problems with the planning, problems with the tools used,\ldots\\
The main risk is that the optimizations proposed are not useful in a practical environment. 

\section{Environmental dimension}
\subsection{PPP}
The estimated electric usage for this project can be found in this table \ref{OtherResources}. The estimation has been done with this expression: $E=\frac{W}{h} \times T$. In this project $E=\frac{51W}{1h}\times 450h = 22,950kW$\\\\
It is hard to minimize more the impact of this project. Some strategies are turning off the computer when not using it, minimizing the amount of paper used, \ldots\\
Some resources are reused, for example, instead of writing all the functionalities, some C++ libraries will be used.
\subsection{Shelf life}
It is hard to measure the footprint of this project along with all its useful life. It will depend on the success of the project, and how many people will use it. \\\\
Currently SAT problems are executed in SAT-Solvers using some optimizations. As explained before, this problem is NP-Complete which among other things implies that there is no known algorithm which can solve it in polynomial time. In other words, solving SAT is very time and resource expensive. \\
Also, SAT is widely used in many fields. For example, computational complexity, databases, programming languages, artificial intelligence and system verification. This translates into a big electricity consumption and a huge footprint. For example, the MareNostrum\cite{MareNostrum} supercomputer spends 1,3MW/year. \\\\
This project purposes more optimizations to reduce the execution time. Even if these optimisations are small, because SAT is widely used, it could have a huge impact. It will have a positive impact because it will reduce the total $CO^2$ emissions released by the computers used to solve them.

\subsection{Risks} 
The footprint of this project could be worst than expected if the development of it is extended.

\section{Social dimension}
\subsection{PPP}
This first stage of the project, GEP, will improve my management and planning skills, my English abilities, how to document and budget projects.\\
The other stages will expand my knowledge about informatics and the opportunity to put in practice a lot of skills developed during this degree. \\
Finally, my ability to present in front of people and defend the work done during these months.

\subsection{Shelf life}
This project will improve a lot of fields because SAT-Solvers are widely used. For example, Planners, Artificial Intelligence, \ldots \ which can have an unpredictable impact in the life of people.\\\\
Currently this problem is solved using other techniques. The solution that this project purposes is an addition to them (it is not exclusive). There is a real need for this type of projects because as said previously, SAT is an NP-Complete problem therefore any improvement on this field will reduce the hardness of the problem with all the consequences this implies.
\subsection{Risks}
The only negative impact that this project can have is not being used. In this case, it will not be used and the society will remain unchanged.

