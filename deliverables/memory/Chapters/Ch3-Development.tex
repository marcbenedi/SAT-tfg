\chapter{Development} 
\label{Chapter3}

\section{Initial Stage} %TODO: Buscarli un nom mes atractiu

Agafar codi actual i "acabarlo"
%TODO: Pensar si en algun punt posar larquitectura del codi base
Google tests
Comencar a fer unit testing
Makefile

Analitzar requisits
Preparar editor


\section{Objective 1: Pseudo-boolean Minimization}

Problema:
Voliem crear una capa per expressar problemes de pseudo minimizatcio pseudo booleana en el nostre software. 
Expressar facil i clarament.

Possibles solucions:
Crear des de zero tot el codi. Requereix molt de temps i impossible de millorar PBLIB.

Utilitzar PBLIB. una llibreria que ja existeix i coneguda dins del secotor. Implementa molts encoding i es una bona eina.

Es tria utilitzar PBLib perque sera menys costos en temps i la solucio sera de millor qualitat.

Pla per executar la solucio:

Planning:

Llegir sobre PBLib i com funciona (adjuntar link als papers)
Compilar-lo, jugar una miqueta, incorporarlo al makefile
Mirar les part mes rellevants del codi i entendre com funcionen.
Estudiar els requeriments que volem i veure si pblib els pot oferir
Dissenyar larquitectura
%TODO adjuntar larquitectura de la primera part.

Development and TDD:

Es va utilitzar metodologia TDD per tant:
com que teniem pensades les funcionalitats que voliem, esolliem la primera i escribiem els tests per aquesta. Despres el codi per passar aquests tests, refactor i passar a la seguent specification.
Design tests for PBFormula
Implement PBFormula
Design tests for PBConstraint
Implement PBConstraing
Design tests for PBMin
Implement PBMin
Design tests for PBMin binary search
Implement binary search
Design tests for PBMin linear search
Implement linear search

Finalization:
Escriure uns quants tests mes, els quals han permes trobar bugs.
%TODO: parlar una de les estrategies de cerca, el perque sutilitza linear search p.e

\section{Objective 2: Timeout}

Problema:
els problemes poden tardar molt en executarse (dir una mesura de temps, algun exemple potser, ...) i per tant shauria de poder parar si ja no interessa.

Possibles solucions:
Implementar timeouts per parar lexecucio
timeout per tot el bucle senser, timeout per cada execucio del minisat

Pla per executar la solucio:

Planning:
buscar informacio sobre estrategies de timout
no vaig trobar molta info (mirar el Diary per mes info)
es van acabar pensant els dos timeout anomenats abans

A l'hora de pensar en larquitectura es va haver de pensar com implementar els timeouts. Vaig pensar en threads o processos fill. Es va acabar triant threads perque comparteixen memoria.

(Al Diary esta explicat que vaig pensant)

Explicar larquitectura 
%TODO adjuntar arquitectura
explicar el refactor del codi anterior
explicar els patrons utilitzats

Development and TDD:

explicar la creacio de stubs per fer el testing correctament.


Finalization:


\section{Objective 3: Multi-threading}

Problema:

Possibles solucions:

Pla per executar la solucio:

Planning:

Development and TDD:

Finalization: