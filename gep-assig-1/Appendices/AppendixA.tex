% Appendix A

\chapter{More information} % Main appendix title

\label{AppendixA} % For referencing this appendix elsewhere, use \ref{AppendixA}

\section{Why SAT Solvers use CNF as input format?}
\label{A.1}

There are two mains reasons for this: Equisatisfiability and Computational Complexity. Let us start with the first one: \\\\

Two \emph{Boolean Formulas} are \textbf{equisatisfiable} if and only if both have the same \emph{models}. This may seem the same as equality but it is not because in an equality relationship both \emph{Boolean Formulas} have to have the same variables. \\
This is important because between a \emph{Boolean Formula} and its result from a \emph{CNF} transformation the equisatisfiability is preserved which means that if the \emph{SAT Solver} finds a \emph{model} for the \emph{CNF}, then this \emph{interpretation} will be also a \emph{model} for the original \emph{Boolean Formula}.\\\\

The second reason is computational complexity. Let us have a look at the following table:

\begin{table}[h]
	\centering
	\begin{tabular}{c|c|c|}
		\cline{2-3}
		& DNF & CNF \\ \hline\hline
		\multicolumn{1}{|c||}{TAUT} & NP  & P   \\ \hline
		\multicolumn{1}{|c||}{SAT}  & P   & NP  \\ \hline
	\end{tabular}
	\caption{Complexity of deciding if a \emph{Boolean Formula} is SAT or TAUT depending of its format.}
	\label{my-label}
\end{table}


So as a \emph{Boolean Formula} can be converted into a \emph{CNF} in linear time while preserving equisatisfiability, \emph{SAT Solvers} will use them to target satisfiability.

\section{What is Unit Propagation?}
\label{A.2}
Unit propagation 