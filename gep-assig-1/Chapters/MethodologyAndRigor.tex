\chapter{Methodology and Rigor} % Main chapter title

\label{Chapter4} % Change X to a consecutive number; for referencing this chapter elsewhere, use \ref{ChapterX}

Research is a vast process with no clear path between \emph{a} and \emph{b}. For this, it is important to follow some directions. Methodology will provide some guidelines to avoid possible problems, be more efficient and do the project more manageable. 

\section{Methodology}
The methodology adopted for this project will be Agile\footnote{Methodology based on the on the adaptability in front of any change to improve exit possibilities.}. It is important to clarify that this methodology will not be followed strictly but adapted to this particular case where there is only one developer and all the objectives are well defined. The main characteristics followed from Agile in this project will be:
\begin{itemize}
	\item Short cycles
	\item TDD (Test-Driven Development)
	\item Weekly scrums with the supervisor
\end{itemize}

\section{Tools}
In this chapter the development tools will be introduced. 
\subsection{Git}
%\href{https://git-scm.com/}{Git} is a well known version control system developed by Linus Torvalds\footnote{Linux creator. \href{https://en.wikipedia.org/wiki/Linus_Torvalds}{(more)}}.\\
\href{https://git-scm.com/}{Git} will be used in this project as a Version Control System because it allows to maintain a tracking of all the changes made (commits), and what is more important, return to them at any time. In addition to this, it enforces a short cycle development (because commits are small units of work) and the developer has to document them which matches perfectly with Agile methodology. \href{https://github.com}{GitHub} will be the repository service used.
\subsection{Trello}
\href{https://trello.com}{Trello} is a simple and flexible web board which helps to organize tasks and its state. It will be used in this project to manage tasks and priorities.  
%\subsection{Mendeley}
%\subsection{CLion}

\section{Communication}
Due to my conditions, I'm currently studying abroad in an Erasmus program, all the communication will be made through electronic means. The majority of it will be made using e-mail but if it is necessary a video conference could be done. \\
The minimum communication with the supervisor will be a weekly e-mail report where all the tasks done during this period will be explained. Problems or questions will be also exposed, if any.

\section{Rigor and Validation}
Rigor and Validation for this project is relevant. \\
The surrounding of it, such as \emph{Artificial Intelligence, Planners, Cryptographic Protocols verification, \ldots}, are widely used nowadays and have been becoming more popular lately. This means that this project could have a big repercussion and be used by some professionals. For this, it is important to guarantee the validation and correctness of the project. \\
During the development, TDD will be used to avoid unnecessary code (possible origin of bugs) and assure the correctness of the implementation. It is also possible to formalize and prove all the operations done by the software.\\
Finally, my supervisor could give me orientation and validate, if necessary, the operations done.