\chapter{Introduction} % Main chapter title

\label{Chapter1} % Change X to a consecutive number; for referencing this chapter elsewhere, use \ref{ChapterX}

en aquesta seccio es fara una primera introduccio al entorn del treball

\section{Context}

Explicar que es SAT, la logica, boolean formula, pseudo boolean, cnf , satsolvers, minimization\\
Target audience: logic researchers, planners etc...\\
Users: C++ programmers which works with logic and want to represent their formulas and improve their time\\
Beneficiaries:\\
\\
\textbf{Boolean satisfiability problems} \textit{(SAT from now on)} is the problem of finding a model\footnote{An interpretation which satisfies the formula.} for a boolean formula. In other words, it is the result of evaluating the boolean formula after replacing its variables for \emph{true} or \emph{false}. 
\\
SAT is widely used in Computer Science because it was the first problem proved to be NP-Complete\cite{Cook1971}\footnote{NP and NP-hard.} which allowed a lot of NP\footnote{Nondeterministic polynomial time.} to be reduced to it.

\subsection{What is a Pseudo-Boolean Formula?}
In propositional logic, a boolean formula is defined as following\cite{Lpo}:\\
Let $P$ be a set of predicate symbols like $p,q,r,...$
\begin{itemize}
	\item All predicate symbol of $P$ is a formula.
	\item If $F$ and $G$ are formulae, then $(F \land G)$ and $(F \lor G)$ are formulae to.
	\item If $F$ is a formula, then $(\neg F)$ is a formula.
	\item Nothing else is a formula.
\end{itemize}
This representation has some limitations because it can only express properties which are \emph{true} or \emph{false}.\\


\section{Background}

Explicar el traball d'investigacio fet aquest quadri

\section{Sate-of-art}

Parlar d'alguns papers anteriors i discutirlos una mica per sobre

\section{Motivation}

Explicar el perque d'aquest treball