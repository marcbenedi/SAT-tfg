\chapter{Introduction and Sate-of-art} % Main chapter title

\label{Chapter1} % Change X to a consecutive number; for referencing this chapter elsewhere, use \ref{ChapterX}

\section{Context}

Before explaining the main problem which this project is about, \emph{Pseudo-Boolean Minimization}, it necessary to do a quick introduction into a much wider topic.\\

\textbf{Boolean satisfiability problems} \textit{(SAT from now on)} is the problem of finding a model\footnote{An interpretation which satisfies the formula.} for a \emph{Boolean Formula} (BF from now on). In other words, it is the result of evaluating the \emph{BF} after replacing its variables for \emph{true} or \emph{false}. 
\\
\emph{SAT} is widely used in Computer Science because it was the first problem proved to be NP-Complete\cite{Cook1971}\footnote{NP and NP-Hard.} which allowed a lot of NP\footnote{Nondeterministic polynomial time.} problems be reduced to it.

\subsection{What is a Pseudo-Boolean Formula?}
In propositional logic, a \emph{BF} is defined as following\cite{Lpo}:\\
Let $P$ be a set of predicate symbols like $p,q,r,...$
\begin{itemize}
	\item All predicate symbol of $P$ is a formula.
	\item If $F$ and $G$ are formulae, then $(F \land G)$ and $(F \lor G)$ are formulae to.
	\item If $F$ is a formula, then $(\neg F)$ is a formula.
	\item Nothing else is a formula.
\end{itemize}
This representation has some limitations because it can only express properties which are \emph{true} or \emph{false}.\\

\emph{Pseudo-Boolean Formulas} are functions of the form $f:B^n \rightarrow \mathbb{R}$. For example, the following formula is a \emph{Pseudo-Boolean Formula} (PBF from now on): $3x+5y$. Therefore, \emph{BF} are a special case of \emph{PBF} where the domain is $d=\{0,1\}$.\\

%TODO: potser explicar que son cardinality constraints


\subsection{Pseudo-Boolean formulae minimization}
\emph{PBF minimization} is a well known NP-Hard\footnote{NP-Hard: at least as hard as the hardest problems in NP \href{https://en.wikipedia.org/wiki/NP-hardness}{(more)}}. \\
It does the following:\\
Given a \emph{PBF} of the form $\sum_{i=1}^{n} x_{i}w_{i} \leq k$, where $w_{i},k \in \mathbb{I}$ and $x_{i} \in \{0,1\}$, it tries to find the minimum $k$ which satisfies the constraint.\\

There is a big research on this field, more specifically in encoding \emph{PBF} into \emph{CNF}. In this paper, Hölldobler, Manthey, Steinke\cite{Holldobler}, some relevant \emph{PBF} into \emph{SAT} encodings are explained and a new one is proposed. One of the authors of this paper, Steinke, is also the author of \emph{PBLib}.  

\section{Background}

During the past semester (Q1 2017/2018), under the supervision of \href{https://www.cs.upc.edu/~jordicf/}{Dr. Jordi Cortadella}, I had been developing a C++ library.\\
This tool allows the users to represent \emph{BF} in a C++ program in a intuitive way, do operations between them and convert them into \emph{Binary Decision Diagrams} (BDD from now on). However, the main functionality of this library is the conversion from a \emph{BF} to \emph{CNF}.  \\
As previously explained, \emph{CNF} is a particular type of a \emph{BF}, a conjunction of disjunctions. \emph{CNF} is an important format because it is the standard input for \emph{SAT Solvers}\ref{A.1}.\\
As shown in this paper, \emph{Mitchell, Selman, and Levesque\cite{Mitchell}}, there is a correlation between the number of variables, the number of clauses and the hardness of solving the \emph{CNF}.
\begin{center}
	\includegraphics[width=1\textwidth]{Figures/GraphMitchellSelmanLevesque.png}
	\captionof{figure}{Median number of recursive DP calls for Random 3-SAT formulas, as a function of the ratio of clauses-to-variables. \\Extracted from Mitchell, Selman, and Levesque\cite{Mitchell}}
\end{center}
Therefore, an improvement of the input \emph{CNF} of the \emph{SAT Solver} can reduce a lot the hardness of the problem. \\
This is the main goal of the library, try to reduce the size of the final \emph{CNF} resulting from applying different converting methods on the original \emph{BF}.

%\section{Sate-of-art}


\section{Motivation}

\href{https://www.fib.upc.edu/en/studies/bachelors-degrees/bachelor-degree-informatics-engineering/curriculum/syllabus/LI}{Informatics Logic} is taught in this\footnote{\href{https://www.fib.upc.edu/en/}{Facultat Informàtica de Barcelona}} faculty. In that course I realized how important is \emph{logic} through its lecturer, \href{http://www.lsi.upc.es/~roberto/}{Dr. Robert Nieuwenhuis}, and its activities. \\

In the first coursework we had to code a \emph{SAT Solver} which used \emph{Unit Propagation}.
%\ref{A.2}. 
With this activity I comprehended how hard and substantial is the study of \emph{logic} and all its context. For example, how \emph{logic} is used in Artificial Intelligence and Planners.\\

When the time of deciding the \emph{TFG} arrived	, I contacted my actual supervisor, \href{https://www.cs.upc.edu/~jordicf/}{Dr. Jordi Cortadella}, and he proposed me some topics and ideas. Finally, we agreed on doing this project. \\

The motivation for this project is try to deepen into the topic and contribute on it.

\section{Stakeholders}

In this section the Stakeholders of the project are defined. Stakeholders are entities which are effected, directly or indirectly, by the solution developed in this project. 
\subsection{Target audience}
This tools tools targets all the entities (researchers, companies, \ldots) which works with \emph{PB minimization} and use \emph{SAT Solvers}.
\subsection{Users}
The users will be C++ programmers due this tool is developed in this language.
\subsection{Beneficiaries}
All those entities which works with \emph{PB minimization}. For example AI, SAT Solvers, Planners, \ldots


