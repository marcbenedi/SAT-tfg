% Chapter Template

\chapter{Project Formulation} % Main chapter title

\label{Chapter1} % Change X to a consecutive number; for referencing this chapter elsewhere, use \ref{ChapterX}

%----------------------------------------------------------------------------------------
%	SECTION 1
%----------------------------------------------------------------------------------------

\section{Introduction}

TODO
There is an excellent introduction
(context) that defines the terms
and concepts of the subject under
study. Stakeholders (target audience, users and beneficiaries)
are fullyspecified.
\\
\\
\textbf{Boolean satisfiability problems} \textit{(SAT from now on)} is the problem of finding a model\footnote{An interpretation which satisfies the formula.} for a boolean formula. In other words, it is the result of evaluating the boolean formula after replacing its variables for \emph{true} or \emph{false}. 
\\
SAT is widely used in Computer Science because it was the first problem proved to be NP-Complete[\cite{Cook1971}]\footnote{NP and NP-hard.}. This is the reason why SAT is so popular, because a lot of NP\footnote{Nondeterministic polynomial time.} problems are reduced to it.

%-----------------------------------
%	SUBSECTION 2
%-----------------------------------

\subsection{What is a pseudo-boolean formula?}
Morbi rutrum odio eget arcu adipiscing sodales. Aenean et purus a est pulvinar pellentesque. Cras in elit neque, quis varius elit. Phasellus fringilla, nibh eu tempus venenatis, dolor elit posuere quam, quis adipiscing urna leo nec orci. Sed nec nulla auctor odio aliquet consequat. Ut nec nulla in ante ullamcorper aliquam at sed dolor. Phasellus fermentum magna in augue gravida cursus. Cras sed pretium lorem. Pellentesque eget ornare odio. Proin accumsan, massa viverra cursus pharetra, ipsum nisi lobortis velit, a malesuada dolor lorem eu neque.

%----------------------------------------------------------------------------------------
%	SUBSECTION 3
%----------------------------------------------------------------------------------------

\subsection{What is minimization?}

Sed ullamcorper quam eu nisl interdum at interdum enim egestas. Aliquam placerat justo sed lectus lobortis ut porta nisl porttitor. Vestibulum mi dolor, lacinia molestie gravida at, tempus vitae ligula. Donec eget quam sapien, in viverra eros. Donec pellentesque justo a massa fringilla non vestibulum metus vestibulum. Vestibulum in orci quis felis tempor lacinia. Vivamus ornare ultrices facilisis. Ut hendrerit volutpat vulputate. Morbi condimentum venenatis augue, id porta ipsum vulputate in. Curabitur luctus tempus justo. Vestibulum risus lectus, adipiscing nec condimentum quis, condimentum nec nisl. Aliquam dictum sagittis velit sed iaculis. Morbi tristique augue sit amet nulla pulvinar id facilisis ligula mollis. Nam elit libero, tincidunt ut aliquam at, molestie in quam. Aenean rhoncus vehicula hendrerit.