% Chapter Template

\chapter{Project Formulation} % Main chapter title

\label{Chapter1} % Change X to a consecutive number; for referencing this chapter elsewhere, use \ref{ChapterX}

%----------------------------------------------------------------------------------------
%	SECTION 1
%----------------------------------------------------------------------------------------

\section{Introduction}
\textcolor{red}{TODO
	There is an excellent introduction
	(context) that defines the terms
	and concepts of the subject under
	study. Stakeholders (target audience, users and beneficiaries)
	are fullyspecified.
	\\
	\\}

\textbf{Boolean satisfiability problems} \textit{(SAT from now on)} is the problem of finding a model\footnote{An interpretation which satisfies the formula.} for a boolean formula. In other words, it is the result of evaluating the boolean formula after replacing its variables for \emph{true} or \emph{false}. 
\\
SAT is widely used in Computer Science because it was the first problem proved to be NP-Complete\cite{Cook1971}\footnote{NP and NP-hard.} which allowed a lot of NP\footnote{Nondeterministic polynomial time.} to be reduced to it.

%-----------------------------------
%	SUBSECTION 2
%-----------------------------------

\subsection{What is First Order Logic?}
In propositional logic, a boolean formula is defined as following\cite{Lpo}:\\
Let $P$ be a set of predicate symbols like $p,q,r,...$
\begin{itemize}
	\item All predicate symbol of $P$ is a formula.
	\item If $F$ and $G$ are formulae, then $(F \land G)$ and $(F \lor G)$ are formulae to.
	\item If $F$ is a formula, then $(\neg F)$ is a formula.
	\item Nothing else is a formula.
\end{itemize}
This representation has some limitations because it can only express properties which are \emph{true} or \emph{false}.\\
This limitations can be overtaken using \textbf{First Order Logic} \textit{(FOL from now on)}. A FOL formula is defined as following\cite{Fol}:\\
\begin{itemize}
	\item text.
	\item text.
	\item text.
	\item text
\end{itemize}

%----------------------------------------------------------------------------------------
%	SUBSECTION 3
%----------------------------------------------------------------------------------------

\subsection{What is minimization?}

Sed ullamcorper quam eu nisl interdum at interdum enim egestas. Aliquam placerat justo sed lectus lobortis ut porta nisl porttitor. Vestibulum mi dolor, lacinia molestie gravida at, tempus vitae ligula. Donec eget quam sapien, in viverra eros. Donec pellentesque justo a massa fringilla non vestibulum metus vestibulum. Vestibulum in orci quis felis tempor lacinia. Vivamus ornare ultrices facilisis. Ut hendrerit volutpat vulputate. Morbi condimentum venenatis augue, id porta ipsum vulputate in. Curabitur luctus tempus justo. Vestibulum risus lectus, adipiscing nec condimentum quis, condimentum nec nisl. Aliquam dictum sagittis velit sed iaculis. Morbi tristique augue sit amet nulla pulvinar id facilisis ligula mollis. Nam elit libero, tincidunt ut aliquam at, molestie in quam. Aenean rhoncus vehicula hendrerit.