% Chapter Template

\chapter{Project Formulation} % Main chapter title

\label{Chapter1} % Change X to a consecutive number; for referencing this chapter elsewhere, use \ref{ChapterX}

\section{Introduction}
\textcolor{green}{
	There is an excellent introduction
	(context) that defines the terms
	and concepts of the subject under
	study. Stakeholders (target
	audience, users and beneficiaries)
	are fullyspecified.
	\\
	\\}

\textcolor{green}{There is an excellent literature
	review onthe subject under study:
	previous studies are cited,
	summarised and discussed.
	It is possible to identify the gap in
	the literature that this project
	addresses. Thus, this project is
	fully supported by the literature.
	\\
	\\}


\textbf{Boolean satisfiability problems} \textit{(SAT from now on)} is the problem of finding a model\footnote{An interpretation which satisfies the formula.} for a boolean formula. In other words, it is the result of evaluating the boolean formula after replacing its variables for \emph{true} or \emph{false}. 
\\
SAT is widely used in Computer Science because it was the first problem proved to be NP-Complete\cite{Cook1971}\footnote{NP and NP-hard.} which allowed a lot of NP\footnote{Nondeterministic polynomial time.} to be reduced to it.

\subsection{What is a Pseudo-Boolean Formula?}
In propositional logic, a boolean formula is defined as following\cite{Lpo}:\\
Let $P$ be a set of predicate symbols like $p,q,r,...$
\begin{itemize}
	\item All predicate symbol of $P$ is a formula.
	\item If $F$ and $G$ are formulae, then $(F \land G)$ and $(F \lor G)$ are formulae to.
	\item If $F$ is a formula, then $(\neg F)$ is a formula.
	\item Nothing else is a formula.
\end{itemize}
This representation has some limitations because it can only express properties which are \emph{true} or \emph{false}.\\


\subsection{What is minimization?}


%
%
%

\section{The project}
\textcolor{green}{The objectives of the project are
	clear and well-specified. The
	proposed project is significant
	enough to be considered a TFG.
	\\
	\\}

